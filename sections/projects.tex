%-------------------------------------------------------------------------------
%	SECTION TITLE
%-------------------------------------------------------------------------------
\cvsection{Technical Projects}


%-------------------------------------------------------------------------------
%	CONTENT
%-------------------------------------------------------------------------------
\begin{cvparagraph}
 
\vspace{2.0mm}

\entrytitlestyle{ChuckPad Social}
\begin{justify}
\begin{itemize}[leftmargin=2ex, nosep]
    \renewcommand{\labelitemi}{\bullet}
\vspace{-2.0mm}
\item ChuckPad is a service enabling users to create, curate, and share generated code from client apps like MiniAudicle.
The service is intended to promote education, enthusiasm, and discovery of content in apps like MiniAudicle.
I developed the server code using Ruby and Sinatra and developed an accompanying iOS library with a suite of unit tests. 
\end{itemize}
\end{justify}


\entrytitlestyle{SoundCraft}
\begin{justify}
\begin{itemize}[leftmargin=2ex, nosep]
    \renewcommand{\labelitemi}{\bullet}
\vspace{-2.0mm}
\item Developed and published a framework that enables real-time data gathering from a StarCraft 2 game, allowing for musical interpretation of the game’s internal structure and strategies in novel ways.
Published in the \href{http://www.nime.org/proceedings/2013/nime2013_146.pdf}{\textbf{New Interfaces for Musical Expression (NIME) Conference 2013}}.
\end{itemize}
\end{justify}

\entrytitlestyle{Laptop Orchestra Network Toolkit (LOrkNeT) - Senior Thesis Research at Princeton}
\begin{justify}
\begin{itemize}[leftmargin=2ex, nosep]
    \renewcommand{\labelitemi}{\bullet}
\vspace{-2.0mm}
\item Developed a toolkit to evaluate networking for live computer music performance.
LOrkNeT was used to identify and remedy issues affecting the Princeton Laptop Orchestra.
Learn more at \href{lorknet.cs.princeton.edu}{\underline{lorknet.cs.princeton.edu}}.
\end{itemize}
\end{justify}


\end{cvparagraph}
