%-------------------------------------------------------------------------------
%	SECTION TITLE
%-------------------------------------------------------------------------------
\cvsection{Technical Projects}


%-------------------------------------------------------------------------------
%	CONTENT
%-------------------------------------------------------------------------------
\begin{cvparagraph}


\entrytitlestyle{ChucK Pad}
\begin{quote}
\\\leading{13pt} 
enables users to create, curate, and share code from music programming environments.
The project's goal is to promote education, stoke enthusiasm, reduce barriers to entry, and aid in discovery of content in apps like MiniAudicle and Auraglyph.
The system consists of a \href{https://github.com/markcerqueira/chuckpad-social}{\underline{Ruby and Sinatra server}},
an accompanying \href{https://github.com/markcerqueira/chuckpad-social-ios}{\underline{iOS API library}} with 
\href{https://github.com/markcerqueira/hello-chuckpad}{\underline{a suite of unit tests}},
and a \href{https://github.com/markcerqueira/chuck-renderer}{\underline{Docker image for rendering ChucK patches}}.
\end{quote}

\entrytitlestyle{SoundCraft}
\begin{quote}
\\\leading{13pt}
is a framework that enables real-time data gathering from a StarCraft 2 game, allowing for musical interpretation of the game’s internal structure and strategies in novel ways.
Published in the \href{http://www.nime.org/proceedings/2013/nime2013_146.pdf}{\underline{New Interfaces for Musical Expression (NIME) Conference 2013}}.
A performance piece - \href{https://github.com/markcerqueira/resume/raw/master/publications/gg-music.pdf}{\underline{GG Music}} - was developed using SoundCraft and performed at the \href{https://www.youtube.com/watch?v=WisMhY5BWa4}{\underline{Music and Gaming Concert}} at Stanford University in April 2013 and the 2013 NIME Conference in Daejeon, Korea.
\end{quote}

\entrytitlestyle{Laptop Orchestra Network Toolkit (LOrkNeT)}
\begin{quote}
\\\leading{13pt}
measures and evaluates network conditions for live computer music performance.
LOrkNeT was used to identify and remedy issues affecting the Princeton Laptop Orchestra.
Learn more at \href{lorknet.cs.princeton.edu}{\underline{lorknet.cs.princeton.edu}}.
\end{quote}

\end{cvparagraph}
